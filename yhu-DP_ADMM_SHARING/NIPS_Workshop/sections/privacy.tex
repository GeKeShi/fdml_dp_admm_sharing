\section{Differentially Private ADMM Sharing}
% The data we have is

% \begin{eqnarray*}
% \left[
% % \nonumber % Remove numbering (before each equation)
%   \begin{array}{cccc}
%     \mathcal{D}_1^1 & \mathcal{D}_2^1 & \cdots & \mathcal{D}_M^1 \\
%     \mathcal{D}_1^2 & \mathcal{D}_2^2 &\cdots & \mathcal{D}_M^2 \\
%     \vdots & \vdots & \ddots & \vdots \\
%     \mathcal{D}_1^N & \mathcal{D}_2^N & \cdots & \mathcal{D}_M^N
%   \end{array}
% \right]
% \end{eqnarray*}
% where $\mathcal{D}_m^i\in\mathcal{A}_m\subset\mathbb{R}^{d_m}$, $i=1,\cdots,N, m=1,\cdots,M$.

% Rewrite the object function
% \begin{eqnarray}
% % \nonumber % Remove numbering (before each equation)
%    &&\sum_{m^{\prime}=1}^{M}l_{m^{\prime}}(\sum_{m=1}^{M}\mathcal{D}_mx_m;Y)
%    +\lambda\sum_{m=1}^{M}R_m(x_m)\label{objtotal} \\
%    &&\text{subject~to}~\sum_{m=1}^{M}\mathcal{D}_mx_m-z=0,z\in\mathbb{R}^N,x_m\in\mathbb{R}^{d_m}, \mathcal{D}_m\in\mathbb{R}^{N\times d_m}, m=1,\cdots, M\nonumber
% \end{eqnarray}

% The augmented Lagrangian for object function (\ref{objtotal}) is:
% \begin{eqnarray*}
% % \nonumber % Remove numbering (before each equation)
%    &&\mathcal{L}_{\rho}(x,z,y) \\
%   &&=\sum_{m=1}^{M}l_m(z;Y)+\lambda\sum_{m=1}^{M}R_m(x_m)\\
%   &&~~+\langle y,\sum_{m=1}^{M}\mathcal{D}_mx_m-z\rangle+\frac{\rho}{2}\|\sum_{m=1}^{M}\mathcal{D}_mx_m-z\|^2
% \end{eqnarray*}

%%%In the following, we employ the first order approximation of the with a scalar $l_2$-norm
%%%prox-function:
%%%\begin{eqnarray*}
%%%% \nonumber % Remove numbering (before each equation)
%%%  &&\hat{\mathcal{L}}_{\rho,t}(x_m,\tilde{x}_m^t,z^t,y^t) \\
%%%  &&=\sum_{m=1}^{M}l_m(z^t;Y)+\lambda\sum_{m=1}^{M}R_m(\tilde{x}_m^t)
%%%  +\langle \lambda\sum_{m=1}^{M}R_m^{\prime}(\tilde{x}_m^t),x_m-\tilde{x}_m^t\rangle \\
%%%  &&+\langle y^t,\sum_{m=1}^{M}\mathcal{D}_mx_m-z^t\rangle+\frac{\rho}{2}\|\sum_{m=1}^{M}
%%%  \mathcal{D}_mx_m-z\|^2+\frac{\|x_m-\tilde{x}_m^t\|^2}{2\eta_m^{t+1}}.
%%%\end{eqnarray*}
%%%where $\eta_m^{t+1}\in\mathbb{R}$ is a time varying step size and decreases with increasing $k$.

Differential privacy \cite{dwork2014algorithmic,zhou2010security} is a notion that ensures a strong guarantee for data privacy. The intuition is to keep the query results from a dataset relatively close if one of the entries in the dataset changes, by adding some well designed random noise into the query, so that little information on the raw data can be inferred from the query. Formally, the definition of differential privacy is given in Definition~\ref{defi:DP}.
% The novelty of our proof is that we provide a differential privacy guarantee of
% $\mathcal{D}_m x^{t+1}_m$ --- the value shared to other parties, instead of $x^{t+1}_m$ itself.
\begin{defi}\label{defi:DP}
A randomized algorithm $\mathcal{M}$ is $(\varepsilon, \delta)-$differentially private if for all $S\subseteq\text{range}(\mathcal{M})$, and for all $x$ and $y$, such that $|x-y|_1\le 1$, we have
\begin{align}
\text{Pr}(\mathcal{M}(x)\in S)\le \exp(\varepsilon)\text{Pr}(\mathcal{M}(y) \in S)+\delta.
\end{align}

\end{defi}
%%%\begin{defi}
%%%$l_1-$sensitivity of function $f$ is
%%%\begin{align}
%%%\Delta f = \underset{\|x-y\|_1\le1}{\max}\|f(x)-f(y)\|_1.
%%%\end{align}
%%%\end{defi}
Definition~\ref{defi:DP} provides a strong guarantee for privacy, where even if most entries of a dataset are leaked, little information about the remaining data can be inferred from the randomized output. Specifically, when $\varepsilon$ is small, $\exp(\varepsilon)$ is approximately $1+\varepsilon$. Here $x$ and $y$ denote two possible instances of some dataset. $|x-y|_1\le 1$ means that even if most of the data entries but one are leaked, the difference between the randomized outputs of $x$ and $y$ is at most $\varepsilon$ no matter what value the remaining single entry takes, preventing any adversary from inferring the value of that remaining entry. Moreover, $\delta$ allows the possibility that the above $\varepsilon$-guarantee may fail with probability $\delta$.

In our ADMM algorithm, the shared messages $\{\mathcal{D}_mx_m^{t+1}\}_{t=0,1,\cdots,T-1}$ may reveal sensitive information from the data entry in $D_m$ of Party $m$. We perturb the shared value $\mathcal{D}_mx^{t+1}_m$ in Algorithm~\ref{alg:ADMM_sharing} with a carefully designed random noise to provide differential privacy. The resulted perturbed ADMM sharing algorithm is the following updates:
\begin{align}
% \nonumber % Remove numbering (before each equation)
  &x_{m}^{t+1}:=\underset{x_m\in X_m}{\text{argmin}}\quad\lambda R_m(x_m) + \langle y^t, \mathcal{D}_mx_m\rangle\nonumber\\
  &\quad\quad\quad\quad\quad\quad\quad + \frac{\rho}{2}\big\|\sum_{\substack{k=1,~k\neq m}}^{M}\mathcal{D}_k\tilde{x}_k^{t} + \mathcal{D}_mx_m - z^t\big\|^2, \nonumber\\&\hspace*{36pt} m=1,\cdots,M
%%%  \arg\min_{x_m}\hat{\mathcal{L}}_{\rho,t}(x_m,\tilde{x}_m^t,z^t,y^t).
  \nonumber\\
  &\xi_m^{t+1}:=\mathcal{N}(0,\sigma_{m,t+1}^2
  (\mathcal{D}_m^\top\mathcal{D}_m)^{-1}
  )\nonumber\\
  &\tilde{x}_m^{t+1}:= x_m^{t+1}+\xi_m^{t+1}\label{admmstepsdp}\\
  &z^{t+1}:=\underset{z}{\text{argmin}}\quad l(z)  - \langle y^t, z \rangle + \frac{\rho}{2} \big\|\sum_{m=1}^{M}\mathcal{D}_m\tilde{x}_m^{t+1} - z\big\|^2\nonumber\\
&y^{t+1}:=y^t + \rho\big(\sum_{m=1}^{M}\mathcal{D}_m\tilde{x}_m^{t+1} - z^{t+1}\big). \nonumber
\end{align}
%\textbf{where $x_{m,\textcolor{red}{\mathcal{D}_m}}^{t+1}$ is to emphasize that $x_{m}^{t+1}$
%is relied on data $\textcolor{red}{\mathcal{D}_m}$, however, in the following, with a slight abuse of notation, we will write $x_{m,\textcolor{red}{\mathcal{D}_m}}^{t+1}$ as $x_{m}^{t+1}$ in
%the case of no ambiguity.}
In the remaining part of this section, we demonstrate that (\ref{admmstepsdp}) guarantees $(\varepsilon, \delta)$~differential privacy with outputs $\{\mathcal{D}_m\tilde{x}_m^{t+1}\}_{t=0,1,\cdots,T-1}$ for some carefully selected $\sigma_{m,t+1}$. Beside Assumption~\ref{theo:assumptions_pri}, we introduce two sets of assumptions widely used by the literature.
% workshop change
\begin{assume}\label{theo:assumptions_pri}
% \newline
\begin{enumerate}
    \item There exists a positive constant $L>0$ such that
        \[
            \|\nabla l(x)-\nabla l(z)\| \le L\|x-z\|\quad \forall x, z.
        \]
        Moreover, for all $m\in\{1,2,\cdots,M\}$, $X_m$'s are closed convex sets; each $\mathcal{D}_m$ is of full column rank so that the minimum eigenvalue $\sigma_{\text{min}}(\mathcal{D}_m^\top \mathcal{D}_m)$ of matrix $\mathcal{D}_m^\top \mathcal{D}_m$ is positive.\label{item:assum_1_pri}
    \item The penalty parameter $\rho$ is chosen large enough such that
    \begin{enumerate}
        \item each $x_m$ subproblem~\eqref{eq:pal_algo_x} as well as the $z$ subproblem~\eqref{eq:pal_algo_z} is strongly convex, with modulus $\{\gamma_m(\rho)\}_{m=1}^M$ and $\gamma(\rho)$, respectively. \label{item:asusum_2_1_pri}
        \item $\gamma_m(\rho)\ge 2\sigma_{\text{max}}(\mathcal{D}_m^\top \mathcal{D}_m), \forall m$, where $\sigma_{\text{max}}(\mathcal{D}_m^\top \mathcal{D}_m)$ is the maximum eigenvalue for matrix $\mathcal{D}_m^\top \mathcal{D}_m$.
        \item$\rho\gamma(\rho)>2L^2$ and $\rho\ge L$.
    \end{enumerate}
    \label{item:assum_2_pri}
    \item The objective function $l\left(\sum_{m=1}^{M} \mathcal{D}_mx_m\right) + \lambda\sum_{m=1}^{M} R_m(x_m)$ in Problem~\ref{eq:analysis_problem} is lower bounded over $\Pi_{m=1}^MX_m$ and we denote the lower bound as $\underline{f}$.\label{item:assum_3_pri}
    \item $R_m$ is either smooth nonconvex or convex (possibly nonsmooth). For the former case, there exists $L_m>0$ such that $\|\nabla R_m(x_m) - \nabla R_m(z_m)\|\le L_m\|x_m-z_m\|$ for all $x_m, z_m\in X_m$.\label{item:assum_4_pri}
\end{enumerate}
\end{assume}

Specifically, \ref{item:assum_1_pri}, \ref{item:assum_3_pri} and \ref{item:assum_4_pri} in Assumptions~\ref{theo:assumptions_pri} are common settings in the literature. Assumptions~\ref{theo:assumptions_pri}.\ref{item:assum_2_pri} is achievable if the $\rho$ is chosen large enough.

\begin{assume}\label{theo:assumptions_pri_added}
  \begin{enumerate}
    \item The feasible set $\{x,y\}$ and the dual variable $z$ are bounded; their $l_2$ norms have an upper bound $b_1$.\label{item:assum_5_pri}
    \item The regularizer $R_m(\cdot)$ is doubly differentiable
    with $|R_m^{\prime\prime}(\cdot)|\leq c_1$, where $c_1$ is a finite constant.\label{item:assum_6_pri}
    \item Each row of $\mathcal{D}_m$ is normalized and has an $l_2$ norm of 1.\label{item:assum_7_pri}
  \end{enumerate}
\end{assume}
Note that Assumption \ref{theo:assumptions_pri_added}.\ref{item:assum_5_pri} is adopted in \cite{sarwate2013signal} and \cite{wang2019global}. Assumption \ref{theo:assumptions_pri_added}.\ref{item:assum_6_pri} comes from \cite{zhang2017dynamic} and Assumption \ref{theo:assumptions_pri_added}.\ref{item:assum_7_pri} comes from \cite{zhang2017dynamic}
and \cite{sarwate2013signal}. As a typical method in differential privacy analysis, we first study the $l_2$ sensitivity of
$\mathcal{D}_mx_m^{t+1}$, which is defined by:
\begin{defi}
The $l_2$-norm sensitivity of $\mathcal{D}_mx_m^{t+1}$ is defined by:
  \begin{eqnarray*}
  % \nonumber % Remove numbering (before each equation)
\Delta_{m,2}=\max_{\substack{\mathcal{D}_m,D_m^{\prime}\\
\|\mathcal{D}_m-D_m^{\prime}\|\leq1
}}\|\mathcal{D}_mx_{m,\mathcal{D}_m}^{t+1}
-\mathcal{D}_m^{\prime}x_{m,\mathcal{D}_m^{\prime}}^{t+1}\|.
  \end{eqnarray*}
  where $\mathcal{D}_m$ and $\mathcal{D}_m^{\prime}$ are two neighbouring datasets differing in 
  only one feature column, and 
  $x_{m,\mathcal{D}_m}^{t+1}$ is the $x_m^{t+1}$ derived from the first line of equation 
  (\ref{admmstepsdp}) under dataset $\mathcal{D}_m$.
\end{defi}
We have Lemma~\ref{lemma:privacy} state the upper bound of the $l_2$-norm sensitivity of $\mathcal{D}_mx_m^{t+1}$.
\begin{lemma}
\label{lemma:privacy}
  Assume that Assumption~\ref{theo:assumptions_pri} and Assumption~\ref{theo:assumptions_pri_added} hold.
  %  the $l_2$-norm sensitivity of $x_m^{t+1}$ is defined by:
%  \begin{eqnarray*}
%  % \nonumber % Remove numbering (before each equation)
%\Delta_{m,2}=\max_{\substack{\mathcal{D}_m,D_m^{\prime}\\
%\|\mathcal{D}_m-D_m^{\prime}\|\leq1
%}}\|\mathcal{D}_mx_{m,\mathcal{D}_m}^{t+1}
%-\mathcal{D}_m^{\prime}x_{m,\mathcal{D}_m^{\prime}}^{t+1}\|
%  \end{eqnarray*}
Then the $l_2$-norm sensitivity of $\mathcal{D}_mx_{m,\mathcal{D}_m}^{t+1}$ is upper bounded by $\mathbb{C}=\frac{3}{d_m\rho}\left[\lambda c_1+(1+M\rho)b_1\right]$.
\end{lemma}

We have Theorem~\ref{theo:DP} for differential privacy guarantee in each iteration.
\begin{theorem}\label{theo:DP}
  Assume assumptions \ref{theo:assumptions_pri_added}.\ref{item:assum_1_pri}-\ref{theo:assumptions_pri_added}.\ref{item:assum_7_pri} hold and $\mathbb{C}$ is the 
  upper bound of $\Delta_{m,2}$. Let $\varepsilon\in(0,1]$ be an arbitrary constant and let
  $\mathcal{D}_m\xi_m^{t+1}$ be sampled from zero-mean Gaussian distribution with variance $\sigma_{m,t+1}^2$,
  where
  \begin{eqnarray*}
  % \nonumber % Remove numbering (before each equation)
    \sigma_{m,t+1}=\frac{\sqrt{2\text{ln}(1.25/\delta)}\mathbb{C}}{\varepsilon}.
  \end{eqnarray*}
  Then each iteration guarantees $(\varepsilon,\delta)$-differential privacy. Specifically,
  for any neighboring datasets $\mathcal{D}_m$ and $\mathcal{D}_m^{\prime}$, for any output
  $\mathcal{D}_m\tilde{x}_{m,\mathcal{D}_m}^{t+1}$ and $\mathcal{D}_m^{\prime}\tilde{x}_{m,\mathcal{D}_m^{\prime}}^{t+1}$, the following inequality always holds:
  \begin{eqnarray*}
  % \nonumber % Remove numbering (before each equation)
    P(\mathcal{D}_m\tilde{x}_{m,\mathcal{D}_m}^{t+1}|\mathcal{D}_m)\leq e^{\varepsilon}
    P(\mathcal{D}_m^{\prime}\tilde{x}_{m,\mathcal{D}_m^{\prime}}^{t+1}|\mathcal{D}_m^{\prime})
    +\delta.
  \end{eqnarray*}
\end{theorem}

With an application of the composition theory in \cite{dwork2014algorithmic}, we come to a result stating the overall privacy guarantee for the training procedure.
\begin{coro}\label{theorem:overall_privacy}
  For any $\delta^{\prime}>0$, the algorithm described in \eqref{admmstepsdp} satisfies $(\varepsilon^{\prime}, T\delta+\delta^{\prime})-$differential privacy within $T$ epochs of updates, where
  \begin{equation}
    \varepsilon^{\prime}=\sqrt{2T\text{ln}(1/\delta^{\prime})}\varepsilon+T\varepsilon(e^\varepsilon - 1).
  \end{equation}
\end{coro}

Without surprise, the overall differential privacy guarantee may drop dramatically if the number of epochs $T$ grows to a large value, since the number of exposed results grows linearly in $T$. However, as we will show in the experiments, the ADMM-sharing algorithm converges fast, taking much fewer epochs to converge than SGD when the number of features is relatively large. Therefore, it is of great advantage to use ADMM sharing for wide features as compared to SGD or Frank-Wolfe algorithms. When $T$ is confined to less than 20, the risk of privacy loss is also confined.
